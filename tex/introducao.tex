\chapter[Introdução]{Introdução}\label{cap_intro}

Este documento e seu código-fonte são exemplos de referência de uso da classe
\textsf{abntex2} e do pacote \textsf{abntex2cite}. O documento
exemplifica a elaboração de trabalho acadêmico (tese, dissertação e outros do
gênero) produzido conforme a ABNT NBR 14724:2011 \emph{Informação e documentação
- Trabalhos acadêmicos - Apresentação}.

Salientamos que este documento é uma customização específica do modelo canônico
disponibilizado em \verb"http://www.abntex.net.br/" para atender à produção de
dissertações e teses do Programa de Pós-Graduação em Modelagem Computacional do
Laboratório Nacional de Computação Científica.

A expressão ``Modelo Canônico'' é utilizada para indicar que \abnTeX\ não é
modelo específico de nenhuma universidade ou instituição, mas que implementa tão
somente os requisitos das normas da ABNT. Uma lista completa das normas
observadas pelo \abnTeX\ é apresentada em \citeonline{abntex2classe}. Além disso,
este documento deve ser utilizado como complemento dos manuais do \abnTeX\
\cite{abntex2classe,abntex2cite,abntex2cite-alf} e da classe \textsf{memoir}
\cite{memoir}.

A equipe de customização deste modelo de dissertação/tese elaborou este documento
de modo que você possa gerar as páginas-modelo do seu trabalho tais como, folha de
rosto, folha com a ficha catalográfica, folha de aprovação, folha com dedicatória,
entre outras, a partir do arquivo principal \verb"tese_lncc.tex". Atente-se aos
campos de preenchimento e utilize o símbolo de porcentagem $\%$ para ativar/desativar
uma linha de interesse. Para utilizar este modelo, o arquivo \verb"abntex2lncc.sty" deve
estar na mesma pasta do arquivo \verb"tese_lncc.tex".

Apesar do exposto no parágrafo anterior, a equipe de customização sugere que você não
escreva o texto de sua dissertação/tese no arquivo principal. Ao invés disto, opte por
criar arquivos \verb".tex" separados, um para cada capítulo, salvando-os na pasta
\verb"capitulos" e chamando-os no arquivo principal via o comando
\verb"\include{capitulos/nome_do_arquivo}". Faça o mesmo se necessitar incluir apêndices
e/ou anexos.

O próximo capítulo ilustra o uso de comandos do \abnTeX e de \LaTeX. No final do capítulo,
algumas referências bibliográficas foram acrescentadas visando exemplificar as diferentes
formas de construí-las. Contudo, a construção de referências bibliográficas é feita em um
arquivo específico o qual chamamos de \verb"bibliografia.bib". É este arquivo que você deve
utilizar para acrescentar as bibliografias que você citará no seu trabalho.

Caso você tenha alguma sugestão para melhorar a customização feita, queira reportar algum
erro encontrado ou necessite de ajuda para utilizar este template, acesse:
\url{https://github.com/equipe-customizacao-tese-lncc/tese_lncc}.

\vspace{20pt}
\begin{flushright}
A equipe de customização.
\end{flushright}